\documentclass[11pt]{amsart}
\usepackage[abbreviate=false,sorting=none,doi=false]{biblatex}
\addbibresource{refs.bib}

\usepackage{xspace}
\usepackage{amsmath}
\usepackage{amsfonts}
\usepackage{latexsym}
\usepackage{graphicx}
\usepackage{bm}
\usepackage{siunitx}

%
% =======================================================================
% Formatting tools
% =======================================================================
% Gentle spacing after macros
% Reference: The LaTeX Companion, p.50
\newcommand{\Real}{\mathbb R}
\newcommand{\eg}{e.g.,\xspace}
\newcommand{\ie}{i.e.,\xspace}
\newcommand{\etc}{etc.\@\xspace}
\newcommand{\etal}{et~al.\@\xspace}
%
%
% =======================================================================
% Common Mathematics
% =======================================================================
\newcommand{\vect}[1]{\boldsymbol{#1}}
\newcommand{\mat}[1]{\boldsymbol{#1}}
\newcommand{\transp}[1]{{#1}^{\ensuremath{\mathsf{T}}}}
%
% =======================================================================
% Math Operators
% =======================================================================
\DeclareMathOperator{\erfc}{erfc}
\DeclareMathOperator{\spanop}{span}
\DeclareMathOperator{\rank}{rank}
\DeclareMathOperator{\tr}{tr}
\DeclareMathOperator{\diag}{diag}
%
\newcommand{\grad}{\nabla}
\newcommand{\dive}{\nabla\cdot}
\newcommand{\curl}{\nabla\times}
\newcommand{\abs}[1]{\left| #1\right|}
\newcommand{\norm}[1]{\left\| #1\right\|}
\newcommand{\fracpar}[2]{\frac{\partial #1}{\partial #2}}
\newcommand{\Fcal}{\mathcal{F}}
\newcommand{\Ccal}{\mathcal{C}}
\let\code\texttt
\newcommand{\ddiv}{\code{div}\xspace}%
\newcommand{\dgrad}{\code{grad}\xspace}%

\newcommand{\nC}{{n_{\Ccal}}}%
\newcommand{\nF}{{n_{\Fcal}}}%


\setlength{\parindent}{0cm}
\setlength{\parskip}{5mm}

\begin{document}

\title[velocity approximation]{Cell approximation of the velocity}

\maketitle

\begin{abstract}
  In this document, we describe how the velocity is approximated on cells.
\end{abstract}


Our starting point is a mixed finite volume method to solve
\begin{subequations}
  \label{eq:contgov}
  \begin{align}
    \label{eq:contgov1}
    \vect{u} &= - \mat{K}\grad p,\\
    \dive \vect{u} &= q
  \end{align}
\end{subequations}
We denote by $\Ccal=\{1,\ldots,\nC\}$ the set of cells and $\Fcal=\{1,\ldots,\nF\}$ the set of
faces. The numbers $\nC$ and $nF$ denote the number of cells and faces, respectively. For a given
cell $c$, we denote $\Fcal_c$ the set of the faces of $c$. Since we have a mixed method, the
integrated fluxes are given as our numerical unknowns,
\begin{equation}
  \label{eq:defuf}
  u_f = \int_{f}\vect{u}\cdot\vect{n}_f\,dx
\end{equation}
Since we have a finite volume method, the discrete divergence operator is the mapping from
$\Real^\nF$ to $\Real^\nC$ given by
\begin{equation}
  \label{eq:defdiv}
  \ddiv(u)(c) = \sum_{f\in\Fcal_c}u_f
\end{equation}
for any $c\in\Ccal$ and $u\in\Real^{\nF}$. The governing equations, either they are discrete or
continuous as in \eqref{eq:contgov}, can be obtained using a variational principle. Indeed
\eqref{eq:contgov} are equivalent to minimize the dissipated energy
\begin{equation}
  \label{eq:totenergy}
  \int_{\Omega} \vect{u}\cdot\mat{K}^{-1}\vect{u}
\end{equation}
given the volume preservation constraint $\dive\vect{u} = q $. The pressure $p$ which appears in
\eqref{eq:contgov1} is then a Lagrange multiplier. At the discrete level, the volume preservation
constraint is already defined through the definition of the discrete divergence operator. It remains
to approximate the energy. This approximation is done cell-wise, by using the faces degrees of
freedom given by the fluxes $u_f$, and by introducing a matrix $M$ for each cell $c\in\Ccal$ such
that
\begin{equation}
  \label{eq:defM}
  \sum_{f\in \Fcal_c} u_f M_{f,f'} v_{f'} \approx \int_{c} \vect{u}\cdot\mat{K}^{-1}\vect{v}\,dx
\end{equation}
In the remaining, we will only consider a given cell $c$ so that we will sometimes drop the
sub-indices and denote
\begin{equation*}
  \sum_{f\in \Fcal_c} u_f M_{f,f'} v_{f'} = u^t\vect{M}v
\end{equation*}

A necessary condition for the method to be convergent is that the approximation is
\emph{consistent}. To define this concept, we remark first that, if $\vect{u}$ is a constant vector,
then we can always find $a$ such that
\begin{equation}
  \label{eq:formu}
  \vect{u} = \mat{K}\grad(\vect{a}\cdot(\vect{x} - \vect{x}_c)),
\end{equation}
for $a=K^{-1}\vect{u}$ and where $\vect{x}_c$ denotes the centroid,
\begin{equation}
  \label{eq:defcentroid}
  \vect{x}_c = \frac{1}{\text{vol}(c)}\int_c \vect{x}\,dx.
\end{equation}
Then, using integration by part, we have
\begin{align}
  \label{eq:intbpart}
  \int_c\vect{u}^t\mat{K}^{-1}\vect{v}\,dx &=  \int_c\grad(\vect{a}\cdot(\vect{x} - \vect{x}_c))\cdot\vect{v}\,dx\\
  \notag                                           &= \int_c\dive((\vect{a}\cdot(\vect{x} - \vect{x}_c))\vect{v})\,dx - \int_c(\vect{a}\cdot(\vect{x} - \vect{x}_c))\dive\vect{v}\,dx\\
  \notag                                           &= \int_{\partial c}(\vect{a}\cdot(\vect{x} - \vect{x}_c))(\vect{n}\cdot\vect{v})\,dx - \int_c(\vect{a}\cdot(\vect{x} - \vect{x}_c))\dive\vect{v}\,dx,
\end{align}
for any vector field $\vect{v}$. By definition, we say that the matrix $M$ gives a consistent
approximation if for any constant vector field $\vect{u}$ and any $\vect{v}$ such that
\begin{subequations}
  \label{eq:condv}
  \begin{align}
&\vect{v} \text{ is constant on each face } f,  \\
&\dive{\vect{v}} \text{ is constant in the cell } c,
  \end{align}
\end{subequations}
we have
\begin{equation}
  \label{eq:consrel}
  \int_c\vect{u}\cdot\mat{K}^{-1}\vect{v}\,dx = u^t\mat{M}v,
\end{equation}
where $u$ and $v$ are the vectors of face-values defined from $\vect{u}$ and $\vect{v}$ as in
\eqref{eq:defuf}. When $v$ satisfy the conditions \eqref{eq:condv}, the expression
\eqref{eq:intbpart} can be simplified as the divergence term cancels and we get
\begin{equation}
 \label{eq:intbpart2}
 \int_c\vect{u}\cdot\mat{K}^{-1}\vect{v}\,dx  = \sum_{f\in\Fcal_c}(\vect{a}\cdot \vect{c}_f)v_f
\end{equation}
where
\begin{equation}
  \label{eq:defcf}
  \vect{c_f} = \int_{f}(\vect{x} - \vect{x}_c)\,dx.
\end{equation}
We introduce the matrix $\mat{C}$ where the column are given by the $\vect{c}_f$. Then, the identity
\eqref{eq:intbpart2} can be rewritten as
\begin{equation}
  \label{eq:intbpart3}
 \int_c\vect{u}\cdot\mat{K}^{-1}\vect{v}\,dx  = \vect{a}^t\mat{C}v
\end{equation}
and the consistency relation \eqref{eq:consrel} is
\begin{equation}
  \label{eq:consrelsimp}
  u^t\mat{M}v = \vect{a}^t\mat{C}v.
\end{equation}
Let us now express $u$ in function of $\vect{a}$. We have, by definition,
\begin{equation*}
  u_f = \int_f (\vect{K}\vect{a})\cdot \vect{n}_f\,dx.
\end{equation*}
After introducing the matrix $N$ whose column is made of the $\int_{f} \vect{n}_f\,dx$, we can
rewrite the identity above in the compact form
\begin{equation}
  \label{eq:relau}
  u = \mat{N}^t\mat{K}^{t}\vect{a}.
\end{equation}
Then, the simplified consistency relation \eqref{eq:consrelsimp} yields
\begin{equation*}
  \vect{a}^t\mat{K}\mat{N}\mat{M}v = \vect{a}^t\mat{C}v,
\end{equation*}
which must hold for all $\vect{a}$ and $v$ so that, finally, we have
\begin{equation}
  \label{eq:consrelfin}
  \mat{K}\mat{N}\mat{M} = \mat{C}.
\end{equation}
\printbibliography

\end{document}

%  LocalWords:  variational
