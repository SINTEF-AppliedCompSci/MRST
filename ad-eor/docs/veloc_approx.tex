\documentclass[11pt]{amsart}
\usepackage[abbreviate=false,sorting=none,doi=false]{biblatex}
\addbibresource{refs.bib}

\usepackage{xspace}
\usepackage{amsmath}
\usepackage{amsfonts}
\usepackage{latexsym}
\usepackage{graphicx}
\usepackage{bm}
\usepackage{siunitx}

%
% =======================================================================
% Formatting tools
% =======================================================================
% Gentle spacing after macros
% Reference: The LaTeX Companion, p.50
\newcommand{\Real}{\mathbb R}
\newcommand{\eg}{e.g.,\xspace}
\newcommand{\ie}{i.e.,\xspace}
\newcommand{\etc}{etc.\@\xspace}
\newcommand{\etal}{et~al.\@\xspace}
%
%
% =======================================================================
% Common Mathematics
% =======================================================================
\newcommand{\vect}[1]{\boldsymbol{#1}}
\newcommand{\mat}[1]{\boldsymbol{#1}}
\newcommand{\transp}[1]{{#1}^{\ensuremath{\mathsf{T}}}}
%
% =======================================================================
% Math Operators
% =======================================================================
\DeclareMathOperator{\erfc}{erfc}
\DeclareMathOperator{\spanop}{span}
\DeclareMathOperator{\rank}{rank}
\DeclareMathOperator{\tr}{tr}
\DeclareMathOperator{\diag}{diag}
%
\newcommand{\grad}{\nabla}
\newcommand{\dive}{\nabla\cdot}
\newcommand{\curl}{\nabla\times}
\newcommand{\abs}[1]{\left| #1\right|}
\newcommand{\norm}[1]{\left\| #1\right\|}
\newcommand{\fracpar}[2]{\frac{\partial #1}{\partial #2}}
\newcommand{\Fcal}{\mathcal{F}}
\newcommand{\Ccal}{\mathcal{C}}
\let\code\texttt
\newcommand{\ddiv}{\code{div}\xspace}%
\newcommand{\dgrad}{\code{grad}\xspace}%

\newcommand{\nC}{{n_{\Ccal}}}%
\newcommand{\nF}{{n_{\Fcal}}}%
\newcommand{\id}{\mat{I}}

\setlength{\parindent}{0cm}
\setlength{\parskip}{5mm}

\begin{document}

\title[velocity approximation]{Cell approximation of the velocity}

\maketitle

\begin{abstract}
  In this document, we describe how the velocity is approximated on cells.
\end{abstract}

\textbf{The mimetic framework}

Our starting point is a mixed finite volume method to solve
\begin{subequations}
  \label{eq:contgov}
  \begin{align}
    \label{eq:contgov1}
    \vect{u} &= - \mat{K}\grad p,\\
    \dive \vect{u} &= q
  \end{align}
\end{subequations}
We denote by $\Ccal=\{1,\ldots,\nC\}$ the set of cells and $\Fcal=\{1,\ldots,\nF\}$ the set of
faces. The numbers $\nC$ and $nF$ denote the number of cells and faces, respectively. For a given
cell $c$, we denote $\Fcal_c$ the set of the faces of $c$. Since we have a mixed method, the
integrated fluxes are given as our numerical unknowns,
\begin{equation}
  \label{eq:defuf}
  \hat{u}_f = \int_{f}\vect{u}\cdot\vect{n}_f\,dx
\end{equation}
Since we have a finite volume method, the discrete divergence operator is the mapping from
$\Real^\nF$ to $\Real^\nC$ given by
\begin{equation}
  \label{eq:defdiv}
  \ddiv(\hat{u})(c) = \sum_{f\in\Fcal_c}\hat{u}_f
\end{equation}
for any $c\in\Ccal$ and $u\in\Real^{\nF}$. The governing equations, either they are discrete or
continuous as in \eqref{eq:contgov}, can be obtained using a variational principle. Indeed
\eqref{eq:contgov} are equivalent to minimize the dissipated energy
\begin{equation}
  \label{eq:totenergy}
  \frac12\int_{\Omega} \vect{u}\cdot\mat{K}^{-1}\vect{u}
\end{equation}
given the volume preservation constraint $\dive\vect{u} = q $. The pressure $p$ which appears in
\eqref{eq:contgov1} is then a Lagrange multiplier. At the discrete level, the volume preservation
constraint is already defined through the definition of the discrete divergence operator. It remains
to approximate the energy. This approximation is done cell-wise, by using the faces degrees of
freedom given by the fluxes $\hat{u}_f$, and by introducing a matrix $M$ for each cell $c\in\Ccal$ such
that
\begin{equation}
  \label{eq:defM}
  \sum_{f\in \Fcal_c} \hat{u}_f M_{f,f'} \hat{v}_{f'} \approx \int_{c} \vect{u}\cdot\mat{K}^{-1}\vect{v}\,dx
\end{equation}
In the remaining, we will only consider a given cell $c$ so that we will sometimes drop the
sub-indices and denote
\begin{equation*}
  \sum_{f\in \Fcal_c} \hat{u}_f M_{f,f'} \hat{v}_{f'} = \hat{u}^t\vect{M}\hat{v}
\end{equation*}

\textbf{The consistency condition}

A necessary condition for the method to be convergent is that the approximation is
\emph{consistent}. To define this concept, we remark first that, if $\vect{u}$ is a constant vector,
then we can always find $a$ such that
\begin{equation}
  \label{eq:formu}
  \vect{u} = \mat{K}\grad(\vect{a}\cdot(\vect{x} - \vect{x}_c)),
\end{equation}
for $a=K^{-1}\vect{u}$ and where $\vect{x}_c$ denotes the centroid,
\begin{equation}
  \label{eq:defcentroid}
  \vect{x}_c = \frac{1}{\text{vol}(c)}\int_c \vect{x}\,dx.
\end{equation}
Then, using integration by part, we have
\begin{align}
  \label{eq:intbpart}
  \int_c\vect{u}^t\mat{K}^{-1}\vect{v}\,dx &=  \int_c\grad(\vect{a}\cdot(\vect{x} - \vect{x}_c))\cdot\vect{v}\,dx\\
  \notag                                           &= \int_c\dive((\vect{a}\cdot(\vect{x} - \vect{x}_c))\vect{v})\,dx - \int_c(\vect{a}\cdot(\vect{x} - \vect{x}_c))\dive\vect{v}\,dx\\
  \notag                                           &= \int_{\partial c}(\vect{a}\cdot(\vect{x} - \vect{x}_c))(\vect{n}\cdot\vect{v})\,dx - \int_c(\vect{a}\cdot(\vect{x} - \vect{x}_c))\dive\vect{v}\,dx,
\end{align}
for any vector field $\vect{v}$. By definition, we say that the matrix $M$ gives a consistent
approximation if for any constant vector field $\vect{u}$ and any $\vect{v}$ such that
\begin{subequations}
  \label{eq:condv}
  \begin{align}
&\vect{v} \text{ is constant on each face } f,  \\
&\dive{\vect{v}} \text{ is constant in the cell } c,
  \end{align}
\end{subequations}
we have
\begin{equation}
  \label{eq:consrel}
  \int_c\vect{u}\cdot\mat{K}^{-1}\vect{v}\,dx = \hat{u}^t\mat{M}\hat{v},
\end{equation}
where $\hat{u}$ and $\hat{v}$ are the vectors of face-values defined from $\vect{u}$ and $\vect{v}$
as in \eqref{eq:defuf}. When $v$ satisfy the conditions \eqref{eq:condv}, the expression
\eqref{eq:intbpart} can be simplified as the divergence term cancels and we get
\begin{equation}
 \label{eq:intbpart2}
 \int_c\vect{u}\cdot\mat{K}^{-1}\vect{v}\,dx  = \sum_{f\in\Fcal_c}(\vect{a}\cdot \vect{c}_f)\hat{v}_f
\end{equation}
where
\begin{equation}
  \label{eq:defcf}
  \vect{c_f} = \int_{f}(\vect{x} - \vect{x}_c)\,dx.
\end{equation}
We introduce the matrix $\mat{C}$ where the column are given by the $\vect{c}_f$. Then, the identity
\eqref{eq:intbpart2} can be rewritten as
\begin{equation}
  \label{eq:intbpart3}
 \int_c\vect{u}\cdot\mat{K}^{-1}\vect{v}\,dx  = \vect{a}^t\mat{C}\hat{v}
\end{equation}
and the consistency relation \eqref{eq:consrel} is
\begin{equation}
  \label{eq:consrelsimp}
  \hat{u}^t\mat{M}\hat{v} = \vect{a}^t\mat{C}\hat{v}.
\end{equation}
Let us now express $\hat{u}$ in function of $\vect{a}$. We have, by definition,
\begin{equation*}
  \hat{u}_f = \int_f (\vect{K}\vect{a})\cdot \vect{n}_f\,dx.
\end{equation*}
After introducing the matrix $N$ whose column is made of the $\int_{f} \vect{n}_f\,dx$, we can
rewrite the identity above in the compact form
\begin{equation}
  \label{eq:relau}
  \hat{u} = \mat{N}^t\mat{K}^{t}\vect{a}.
\end{equation}
Then, the simplified consistency relation \eqref{eq:consrelsimp} yields
\begin{equation*}
  \vect{a}^t\mat{K}\mat{N}\mat{M}v = \vect{a}^t\mat{C}\hat{v},
\end{equation*}
which must hold for all $\vect{a}$ and $v$ so that, finally, we have
\begin{equation}
  \label{eq:consrelfin}
  \mat{K}\mat{N}\mat{M} = \mat{C}.
\end{equation}

\textbf{A useful identity}

Let us now derive the following useful identity,
\begin{equation}
  \label{eq:pseudoinv}
  \mat{N}\mat{C}^t = \text{vol}(c)\id,
\end{equation}
which states that, up to a coefficient,  $\mat{N} is $
Equation \eqref{eq:pseudoinv} is the discrete consequence of Stokes theorem
\begin{equation}
  \label{eq:stokes}
  \int_{c}\dive \vect{v}\,dx = \int_{\partial c}\vect{v}\cdot\vect{n}\,dx
\end{equation}
applied to linear mappings, that is $\vect{v}$ of the form $\vect{v}= \mat{A}(\vect{x} - \vect{x}_c)$. Indeed, for
such $\vect{v}$, we have
\begin{equation}
  \label{eq:divetrA}
  \dive\vect{v} =\tr(\mat{A})
\end{equation}
and
\begin{align*}
  \int_{\partial c}\vect{v}\cdot\vect{n}\,dx = \sum_{f}(\mat{A}(\int_{f}(\vect{x} - \vect{x}_c)\,dx)\cdot\vect{n}_f) = \tr(\mat{N}^t\mat{A}\mat{C}) = \tr(\mat{C}\mat{N}^t\mat{A})
\end{align*}
We recall that the bilinear form $\mat{B}\vdots\mat{D} = \tr(\mat{B}^t\mat{D})$ is a scalar product
on square matrices. Hence, the Stokes identity \eqref{eq:stokes} can be rewritten as
\begin{equation*}
  vol(c)\id\vdots\mat{A} = (\mat{N}\mat{C}^t)\vdots\mat{A}.
\end{equation*}
Since the above identity must hold for all $\mat{A}$, \eqref{eq:pseudoinv} follows. Using the
identity \eqref{eq:pseudoinv}, we can verify that the matrix
\begin{equation*}
  \mat{M}_0 = \frac{1}{\text{vol}(c)}\mat{C}^t\mat{K}^{-1}\mat{C}
\end{equation*}
satisfies the consistency relation \eqref{eq:consrelfin}. All the matrices which satisfy the
consistency relation are then given by
\begin{equation}
  \label{eq:genexpM}
  \mat{M} = \mat{M}_0 + \tilde{\mat{M}}
\end{equation}
such that $\mat{N}\tilde{\mat{M}}=0$.

\textbf{The two-point flux approximation}

In a two-point flux approximation, we require that $M$ is diagonal. Then, it is straightforward to
invert it. Let us denote the coefficients on the diagonal by $M_{ff}=\frac{1}{t_f}$. The consistency
relation can then be rewritten as
\begin{equation}
  \label{eq:orthogstat}
  \frac{1}{t_f}\mat{K}\vect{n}_f = \vect{c}_f.
\end{equation}
The statement \eqref{eq:orthogstat} means that $\mat{K}\vect{n}_f$ is parallel to $\vect{c}_f$ for
each face $f$. A grid which satisfies this requirement for all cell is said to be
$\mat{K}$-orthogonal. The coefficient $t_f$ is called the \textit{half-transmissibility} and, after
taking the scalar product with $\vect{c}_f$ on both sides, we get
\begin{equation}
  \label{eq:deftf}
  t_f = \frac{\vect{c}_f^t\mat{K}\vect{n}_f}{\abs{\vect{c}_f}^2}
\end{equation}

\textbf{Approximation of cell-valued velocities}

The problem is the following: Given face-valued fluxes $\hat{u}$, obtain a consistent approximation
of the velocity $\vect{u}$. We start from the consistency condition as stated in \eqref{eq:consrel}
which expresses the fact that our discrete approximation of the energy, by using face-valued fluxes
and the matrix $M$, is exact for a certain class of arguments. The condition must hold for constant
vectors, that is,
\begin{equation}
  \label{eq:constve}
  \hat{u}^t\mat{M}\hat{v} = \text{vol}(c)\vect{u}^t\mat{K}^{-1}\vect{v}.
\end{equation}
We want to find a velocity $\vect{u}$ which satisfies \eqref{eq:constve} for at least all constant
vector $\vect{v}$. Let us consider $\vect{v} = \mat{K}\vect{w}$. We have for each face $f$,
\begin{equation*}
  \hat{v}_f = \int_f\mat{K}\vect{w}\cdot\vect{n}_f\,dx,
\end{equation*}
that is,
\begin{equation*}
  \hat{v} = \mat{N}^t\mat{K}\vect{w}.
\end{equation*}
Hence, we can rewrite \eqref{eq:constve} as
\begin{equation*}
  \vect{u}^t\vect{w} = \left(\frac{1}{\text{vol}(c)}\mat{K}\mat{N}\mat{M}\hat{u}\right)^t\vect{w},
\end{equation*}
and, since this identity must hold for all $\vect{w}$, we get the following expression for the
velocity
\begin{equation*}
  \vect{u} = \frac{1}{\text{vol}(c)} \mat{K}\mat{N}\mat{M}\hat{u}.
\end{equation*}
We use now the consistency relation \eqref{eq:consrelfin} and finally obtain an expression for the cell-valued
velocity as a function of the face-valued fluxes,
\begin{equation}
  \label{eq:velocapprox}
  \vect{u} = \frac{1}{\text{vol}(c)} \mat{C}\hat{u}.
\end{equation}
A simple interpretation of this expression is that the velocity is obtained as a combination of the
centroid vectors $\vect{c}_f$ weighted by the flux $\hat{u}_f$,
\begin{equation}
  \label{eq:simpleinterpvelapp}
  \vect{u} = \frac{1}{\text{vol}(c)} \sum_{f\in\Fcal_c}\hat{u}_f\vect{c}_f.
\end{equation}
Note that, through \eqref{eq:pseudoinv}, we can check that the vector $\vect{u}$ satisfies the
identity
\begin{equation*}
  \mat{N}\vect{u} = \hat{u},
\end{equation*}
which corresponds to \eqref{eq:defuf} in the case of a constant vector $\hat{u}$.


\printbibliography

\end{document}

%  LocalWords:  variational transmissibility
