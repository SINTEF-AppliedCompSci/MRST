\documentclass[a4paper,english, final, 11pt]{article}
\usepackage[a4paper,margin=2.5cm]{geometry}
\renewcommand*\familydefault{\sfdefault}
\usepackage[T1]{fontenc} % Vise norske tegn
\usepackage[utf8]{inputenc} % For å kunne skrive norske tegn
\usepackage{babel} % Tilpasning til norsk
\usepackage{amsmath,amssymb,mathtools} % Ekstra matematikkfunksjoner
\usepackage{hyperref}

%----------------------------------------------------------------------------------------
%	TITLE & AUTHOR
%----------------------------------------------------------------------------------------

\title{TMA4500 Specialization Project Description \\
\textbf{Higher-order Virtual Element Methods for Irregular Grids, with Application to Reservoir Simulation.}}
\author{Øystein Strengehagen Klemetsdal\thanks{%
NTNU, Industrial Mathematics, %
\href{mailto:oysteskl@stud.ntnu.no}{oysteskl@stud.ntnu.no}} \\
Supervisor: Xavier Raynaud\thanks{%
NTNU, Department of Mathematics, and SINTEF ICT %
\href{mailto:xavier.raynaud@sintef.no}{xavier.raynaud@sintef.no}}}

\begin{document}

\maketitle

%%%%%%%%%%%%%%%%%%%%%%%%%%%%%%%%%%%%%%%%%%%%%%%%%%%%%%%%%%%%%%%%%%%%%%%%%%%%%%%%%%%%%%%%%%%%%%%%%%%%%%%%%%%%%
%    INTRODUCTION    INTRODUCTION    INTRODUCTION    INTRODUCTION    INTRODUCTION    INTRODUCTION           %
%%%%%%%%%%%%%%%%%%%%%%%%%%%%%%%%%%%%%%%%%%%%%%%%%%%%%%%%%%%%%%%%%%%%%%%%%%%%%%%%%%%%%%%%%%%%%%%%%%%%%%%%%%%%%
\section{Introduction}
%
Today, the industry standard for petroleum reservoir simulations is the finite volume method
(REF). Recently, however, the virtual element method (VEM) (REF) has gained popularity. The method is
particularly interesting because it facilitates computation on very general meshes, which often is the case
with real-life reservoir simulations.

%%%%%%%%%%%%%%%%%%%%%%%%%%%%%%%%%%%%%%%%%%%%%%%%%%%%%%%%%%%%%%%%%%%%%%%%%%%%%%%%%%%%%%%%%%%%%%%%%%%%%%%%%%%%%
%    VEM    VEM    VEM    VEM    VEM    VEM    VEM    VEM    VEM    VEM    VEM    VEM    VEM    VEM    VEM  %
%%%%%%%%%%%%%%%%%%%%%%%%%%%%%%%%%%%%%%%%%%%%%%%%%%%%%%%%%%%%%%%%%%%%%%%%%%%%%%%%%%%%%%%%%%%%%%%%%%%%%%%%%%%%%
\section{The Virtual Element Method}
%
On each element, the trial and test functions contain all polynomials
of degree less than or equal to $k$, along with functions that are, in general, non-polynomial. Typically,
in finite volume and finite element methods, the local stiffness bilinear form is approximated.
VEM handles the stiffness bilinear form differently: When one of the two entries is a polynomial, the exact
value is calculated. In all other cases, only the right order of magnitude and stability properties are
calculated.

%%%%%%%%%%%%%%%%%%%%%%%%%%%%%%%%%%%%%%%%%%%%%%%%%%%%%%%%%%%%%%%%%%%%%%%%%%%%%%%%%%%%%%%%%%%%%%%%%%%%%%%%%%%%%
%    GOAL    GOAL    GOAL    GOAL    GOAL    GOAL    GOAL    GOAL    GOAL    GOAL    GOAL    GOAL    GOAL   %
%%%%%%%%%%%%%%%%%%%%%%%%%%%%%%%%%%%%%%%%%%%%%%%%%%%%%%%%%%%%%%%%%%%%%%%%%%%%%%%%%%%%%%%%%%%%%%%%%%%%%%%%%%%%%
\section{Goal of the Project}
%
The the open-source MATLAB toolbox Matlab Reservoir Simulation Toolbox (MRST) (REF), developed at SINTEF ICT,
already includes an implementation of a first-order (?) mimetic finite difference discretization (REF),
which is a method closely related to VEM (REF). The goal of this project is then to implement a higher-order
VEM for reservoir simulations, in the perspective of laboratory simulations, rock mechanics, and fracture
rock modelling. In addition, the project will serve as a preliminary for a master thesis on the subject. 

\end{document}